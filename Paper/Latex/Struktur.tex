%%This is a very basic article template.
%%There is just one section and two subsections.
\documentclass[11pt,a4paper,oneside]{article}
\usepackage[ngerman]{babel}
\usepackage[utf8]{inputenc}
\usepackage{verbatim}

%%This is a very basic article template.
%%There is just one section and two subsections.
\documentclass[11pt,a4paper,oneside]{report}
\usepackage[ngerman]{babel}
\usepackage[utf8]{inputenc}

\begin{document}

%\input{./title.tex} \end{document}




\title{Interface gebundene Konstruktion eines Control Sets nach Pivtoraiko and Kelly}
\author{Vincent Dahmen (6689845), Tina Van (6707886)\\ Fakultät Informatik der Universität Hamburg}
\date{...}

\maketitle 

\end{document}

\begin{document}

\maketitle 
\pagebreak

\section{Zusammenfassung}
\input{../Notizen/Summary.txt}

\section*{Einleitung}
\input{../Notizen/Introduction.txt}
\pagebreak

\tableofcontents
\pagebreak

\section{Definitionen der Datenstrukturen}

\subsection{ Control-Set}
\input{../Notizen/2.1.Body.txt}
\subsection{Bewegung}
\input{../Notizen/2.2.Body.txt}
\subsection{Bewegungserzeuger}
\input{../Notizen/2.3.Body.txt}
\subsection{Position}
\input{../Notizen/2.4.Body.txt}
\subsection{Bewegungstunnel}
\input{../Notizen/2.5.Body.txt}

\section{Implementation}
\input{../Notizen/3.0.Body.txt}

\section{ Fazit}
\input{../Notizen/FazitNotizen.txt}

\section{Quellen}
\input{../Notizen/Bibliography.txt}

\section{Appendices}
% link zum repo?
\subsection*{ControlSet.java}
\subsection*{Bewegung.java}
\subsection*{BewegungsErzeuger.java}
\subsection*{Position.java}
\subsection*{NeutralePosition.java}
\subsection*{Bewegungstunnel.java}

\end{document}